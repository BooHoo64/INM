\section{Bewertung und Gewichtung (TiK)}
Abschließend kann gesagt werden, dass die Hochschule Emden/Leer derzeit über kein Informationsmanagement verfügt. Informationen werden zentral gesammelt und wichtige Systeme wie das Y-Laufwerk und die E-Learning Plattform Moodle werden in allen Fachbereichen und in Teilen der Verwaltung eingesetzt. Durch den starken Kooperationsverbund werden zentrale Dienste, wie SpringerLink, WISO und video2brain für die Studierenden und Mitarbeiter zur Verfügung gestellt. 

Im Bereich der Repräsentation von Informationen nach außen verfügt die Hochschule über eine Pressestelle und eine Marketingabteilung. Es existiert eine feste CI-Reglung für alle Abteilungen und Bereiche. 

Durch diverse Arbeitsgruppen ist der Erfahrungs-, Wissens- und Informationsaustausch für wichtige zentrale Bereiche bereits gegeben. Durch die Arbeitsgruppe ZDF, WEB und Moodle werden zentrale Systeme zur Wissenserhaltung und Informationsbereitstellung gepflegt. Dadurch das diese Arbeitsgruppen abteilungsübergreifend agieren besteht auch zwischen den einzelnen Bereichen eine Schnittstelle ohne die autarken Fachbereiche einzuschränken. 

Im Bezug auf Serviceorientierung und IT-Sicherheit lässt sich sagen, dass Single-Sign-On (SSO) für viele Bereiche bereits zum Einsatz kommt (siehe Kapitel 2. \ref{...}). Dies ist ein erster Schritt zum Informationsmanagement. Jedoch fehlt grundsätzlich ein zentrales System für den direkten Zugriff und zur Weiterleitung auf weitere Informationssysteme. In Kapitel 1.x wird bereits beschrieben, dass in Hochschulen, welche ein Informationsmanagement bereits einsetzen, dieses meist im Bereich Immatrikulations- und Prüfungsamt (HIS) angesiedelt ist. 
Neben einem  zentralem System fehlt auf der organisatorischen Seite eine Instanz. Wie in Kapitel 1.x beschrieben findet im klassischen Informationsmanagement für Unternehmen das Management häufig durch einen Chief Information Officer (CIO) statt. In Hochschulen wird dies oft durch DACH Organisationen realisiert. 

Auch wenn die Hochschule Emden/Leer bereits diverse Arbeitsgruppen einsetzt, so ist diese Instanz des Informationsmanagements bisher unbesetzt. Ein Informationsmanagement, wie es in Kapitel 1 beschrieben ist, wird derzeit an der Hochschule nicht praktiziert. 
