\chapter{Einleitung - AW}
\textit{Autor: Andreas Willems}

Informationen stellen für Organisationen zunehmend eine Ressource dar, die es effizient zu nutzen gilt.
Unter anderem bedingt durch den Einzug vernetzter Geräte in alle Lebens- und Unternehmensbereiche kommt es zu einer massiven Flut
an Daten und Informationen, welche durch geschicktes Auswerten und Weiterleiten an die richtigen Empfänger zu einer hohen 
Qualität in der Informationsbereitstellung führen sollen.

Die effiziente Bereitstellung und Verarbeitung von Informationen ist Aufgabe eines Informationsmanagements.

Im Rahmen dieser Arbeit soll das Informationsmanagement an Hochschulen zunächst allgemein betrachtet werden und
anschließenden am Beispiel der Hochschule Emden/Leer im speziellen angewandt werden.

Dabei wird die Hochschule Emden/Leer zunächst auf bereits bestehende Anwendungen des Informationsmanagements hin
untersucht. Anschließend wird unter Berücksichtigung von Erfahrungen an anderen Hochschulen ein Konzept ausgearbeitet,
welches das Informationsmanagement an dieser Hochschule erweitern und optimieren soll.

Das entworfene Konzept wird zuletzt auf seine Umsetzbarkeit sowie auf seine finanziellen und zeitlichen Anforderungen hin
untersucht.

\todo[inline]{Einleitung überarbeiten} 