\subsection{Kostenarten in der IT}
In einer Hochschule ist ein Rechenzentrum für die Aufgaben der IT zuständig.
Der Rechenzentrumsleiter ist in einem Netzwerk, aus leitenden Personen der Hochschule, das Zentrum der personellen IT-Komponenten. Im Bereich der IT einer Hochschule wird im Bereich der direkten Kosten zwischen den Primärkategorien Hard- und Software, operativer Betrieb und Verwaltung differenziert.\footnote{\cite{hansen_business_2009}}

Zusätzlich ist ein Rechenzentrum der Hochschule, nach Interview-Aussage des Rechenzentrumsleiters der Hochschule Emden/Leer, jährlich auf ein bestimmtes Budget festgelegt. Dabei ist die Tabelle \ref{tab_auswahl_IT_kostenarten} zu beachten, welche Kosten budgetiert werden können und welche nicht.

\begin{table}[h!]
	\begin{tabularx}{\textwidth}{|X|X|}
		% Überschriften
		\hline \textbf{Budgetierte Kosten}  &  \textbf{Nicht budgetierte Kosten}\\
		% Zeile 1
		\hline Software-Entwicklung 
			\begin{itemize}
				\item Neuentwicklung und Anpassungen
				\item Personal- und Sachkosten	
			\end{itemize}  
		& Negative Produktivitätseffekte
			\begin{itemize}
				\item Antwort-, Rüst- und Bearbeitungszeit
				\item Motivation
				\item Ergonomie
			\end{itemize} \\ 
		% Zeile 2	
		\hline Kommunikation \begin{itemize}
			\item Netzwerk
			\item Personal- und Sachkosten			
		\end{itemize} & Ausfall \begin{itemize}
			\item geplant
			\item ungeplant
		\end{itemize} \\ 
		% Zeile 3
		\hline Hardware / Software \begin{itemize}
			\item Abschreibung, Miete und Leasing
			\item Entsorgung
			\item Client / Server
			\item Administration	
		\end{itemize} & Endbenutzer \begin{itemize}
			\item Peer-Support (selbst/gegenseitig)
			\item Unproduktives Konfigurieren
			\item Qualifizierung (selbst /gegenseitig)
		\end{itemize}  \\
		% Zeile 4
		\hline Support 
			\begin{itemize}
				\item Help-Desk
				\item Personal-, Sach- und 
				\item Gemeinkosten
				\item Intern/Extern
				\item Schulung Intern/Extern
			\end{itemize} & \\
		% Zeile 5 
		\hline Systembetrieb und Systemmanagement
			\begin{itemize}
				\item Verwaltung
				\item Installation / Optimierung
				\item Instandhaltung
			\end{itemize} & \\
		\hline
	\end{tabularx}
	\caption{Auswahl IT-Kostenarten nach Krcmar}
	\label{tab_auswahl_IT_kostenarten}
\end{table}

Als spezielle IT-Kostenarten werden von Gadatsch und Mayer\footnote{\cite{gadatsch_masterkurs_2014}} aufgelistet:
\begin{table}[h!]
	\begin{tabularx}{\textwidth}{|l|X|}
		% Überschriften
		\hline \textbf{Sekundäre Kostenarten}  &  \textbf{Primäre Kostenarten}\\
		% Zeile 1
		\hline Hardware-Kosten &
		\begin{itemize}
			\item Miete / Leasing
			\item Hardware
			\item Leitungsgebühren
			\item Wartung
		\end{itemize} \\ 
		% Zeile 2	
		\hline Software-Kosten  & 
		\begin{itemize}
			\item Miete / leasing
			\item Software
			\item eigene Entwicklung
			\item Externe Wartung
			\item Beratung
		\end{itemize} \\ 
	% Zeile 3
	\hline Daten-Kosten &
	\begin{itemize}
		\item Beratung
		\item Kauf
	\end{itemize}  \\
	% Zeile 4
	\hline Sonstige IT-Kosten &
		\begin{itemize}
			\item IT-Verbrauchsmaterial
			\item IT-Versicherungen
			\item Beiträge zu Fachverbänden
			\item IT-Fachliteratur
			\item IT-Schulungen
		\end{itemize}\\
	% Zeile 5 
	\hline Innerbetriebliche IT-Leistungsverrechnung & 
	\begin{itemize}
		\item Umlagen
		\item Entwicklungskosten
		\item Benutzerservice
	\end{itemize}\\
	\hline
\end{tabularx}
\caption{Auflistung der speziellen IT-Kosten, nach Gadatsch \& Mayer}
\label{tab_auflistung_spezielle_IT_Kosten}
\end{table}

Die Kostenarten aus den Auflistungen der Tabelle \ref{tab_auswahl_IT_kostenarten} und Tabelle \ref{tab_auflistung_spezielle_IT_Kosten} eignen sich für die Betrachtung der Kosten an einer Hochschule.\footnote{\cite{hansen_business_2009}}

Im Bezug auf die Kostenarten schlägt Krcmar z.B. die TCO-Methode (“Total Cost of Ownership”) als Bewertungstechnik vor, was auch im nächsten Kapitel beschrieben und verwendet wird.\footnote{\cite{krcmar_einfuhrung_2015}} Die TCO-Methode nutzt die Kostenarten, um die wirtschaftlichen Auswirkungen in der IT aufzuzeigen.

Vor allem im Bezug auf Kostenarten und IT wird als Trend ein IT-Controller empfohlen, um eine bessere Wertschöpfung in der IT zu erreichen, was im Blick auf die Kostenart Personal einen wirtschaftlichen Vorteil bewirkt. Der IT-Controller sollte jedoch nicht in einer rein bestimmenden Funktion auftreten, sondern eher als Motivator für mehr Effizienz, Ergonomie und Effektivität, was im Umkehrschluss bessere Arbeitsbedingungen und damit auch mehr Erfolg für Projekte bringt.\footnote{\cite{reim_erfolgsrechnung_2015}}

Gerade in einem solch zentralen Projekt mit einem hohen Anteil an IT-lastigen Themen ist es zumindest empfehlenswert über einen IT-Controller nachzudenken.\footnote{\cite{stratmann_it_2013}} Besonders ist hier auch die höhere Komplexität zu beachten, die im Verlauf der Zeit in der IT der DFG-Referenzprojekte entstanden. Die Projekte werden im Kapitel \ref{section_projekt_beispiele} vorgestellt bzw. genannt.