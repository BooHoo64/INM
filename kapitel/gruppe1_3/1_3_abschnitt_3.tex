\section{Zusammenfassung}
Zusammenfassend lässt sich sagen, dass die betrachteten Hochschulen in der Umsetzung des Informationsmanagements zwar oft im Detail unterschiedliche Konzepte verfolgen, in einigen Punkten aber die Gemeinsamkeiten überwiegen.

So setzen alle betrachteten Hochschulen auf eine gewisse Integration von Rechenzentrum, Mediendiensten und oft auch Bibliothek und Verwaltung unter einer zentralen Dachorganisation, die die unterschiedlichen Bereiche koordiniert und von einem CIO oder einem mit den normalerweise mit dem CIO assoziierten Aufgaben betrauten Ausschuss geleitet wird.

Sehr hoher Wert wird generell auf die Nutzerorientierung gelegt. Oft existieren Instanzen, in deren Aufgabenbereich es explizit fällt, diese Nutzerorientierung zu gewährleisten.

Um die heterogenen Anforderungen einer Universität überschaubar umsetzen zu können, setzen einige der betrachteten Hochschulen auf eine Serviceorientierte Architektur.

Bezüglich der Umsetzung von ITIL herrscht in den meisten betrachteten Hochschulen zwar ein gewisser Wille, dieser reicht jedoch nur selten zu einer umfangreichen praktischen Umsetzung.