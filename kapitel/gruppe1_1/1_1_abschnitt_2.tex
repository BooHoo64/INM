\section{Informationsmanagementmodelle in der Literatur}
In der deutschsprachigen Literatur lassen sich viele verschiedene Arbeiten und Definitionen zum Thema Informationsmanagement finden, die sich zum Teil deutlich voneinander unterscheiden. Im folgenden werden die Modelle und Sichtweisen zum Informationsmanagement von Heinrichs\footnote{Wer und was bin ich?}, Wollnik\footnote{Wer und was bin ich?} und Krcmar\footnote{Wer und was bin ich?} vorgestellt.

\subsection{Informationsmanagement nach Heinrichs}
Lange Zeit stellte das 1987 erschienene Werk\footnote{Welches Werk?} von Heinrich das deutschsprachige Standardwerk im Bereich des Informationsmanagement dar. Entsprechend wurde es auch als Lehrbuch an Hochschulen eingesetzt.\\

Laut Heinrich wird unter Informationsmanagement das “Leitungshandeln (Management) in Unternehmen in Bezug auf Information und Kommunikation” verstanden. Es umfasst alle Führungsaufgaben, die sich mit Information und Kommunikation befassen. Diese Informations- und Kommunikationsaufgaben werden als Informationsfunktion bezeichnet, die den Schwerpunkt des Informationsmanagements darstellt.\\

Das Ziel des Informationsmanagements laut Heinrich ist es, eine Informationsinfrastruktur aufzubauen, die die Verteilung, Produktion und Nutzung vom Informationen zur Aufgabe hat. Die Informationsinfrastruktur dient dazu, das Leistungspotenzial der Informationsfunktion umzusetzen und somit einen optimaler Beitrag zum Unternehmenserfolg zu leisten.\\

Für die Umsetzung der Ziele werden die Aufgaben des Informationsmanagements in drei Ebenen strukturiert.
Die \textbf{strategische} Ebene plant, überwacht uns steuert die Informationsinfrastruktur.
Die \textbf{administrative} Ebene plant, überwacht und steuert die Komponenten der Informationsinfrastruktur (z.B. Anwendungssysteme, Mitarbeiter, Bestand an Daten).
Die \textbf{operative} Ebene umfasst Aufgaben und Nutzung der Informationsinfrastruktur. Mögliche Aktionsfelder für die operative Aufgabenebene stellen den laufenden Betrieb, die Nutzerunterstützung und die Störungsbeseitigung dar.\\

Auf jeder Aufgabenebene werden Methoden, Techniken und Werkzeuge eingesetzt, die die Durchführung der strategischen, administrativen und operativen Aufgaben durchführt und unterstützt. Die Gesamtheit dieser Methoden und Techniken wird von Heinrich als \emph{Information Engineering} bezeichnet.

\subsection{Informationsmanagement nach Wollnik}
Wollnik\footnote{In welchem Werk?} gliedert das Informationsmanagement in drei Ebenen.

Die \textbf{Ebene des Informationseinsatzes} und dessen Management befasst sich mit der Integration von Informationen in Produkte und Dienstleistungen. Des weiteren befasst es sich mit der Erschließung neuer Märkte durch den Einsatz von Informationstechnologie.\\

Die \textbf{Ebene der Informations- und Kommunikationssysteme} stellt die mittlere Managementebene dar. Laut Wollnik bestehen Informationssysteme aus folgenden Elementen/Komponenten: Aufgaben, Informationen, Personen, Geräte, Organisation und Programme. Diese bestimmen die Struktur eines Informationssystems. Die Aufgaben dieser Ebene sind die Festlegung, Erhaltung und Modifikation dieser Strukturen während des Lebenszyklus des Informationssystems.\\

Ein weiteres Handlungsobjekt dieser Ebene sind die Prozesse zur Gestaltung von Informationssystemen, die geplant, organisiert und kontrolliert werden müssen. Diese Ebene stellt das Verbindungsglied zwischen den betrieblichen Aufgaben (Ebene Eins) und der technischen Infrastruktur (Ebene Drei) dar.\\

Die \textbf{Ebene der Informations- und Kommunikationsinfrastruktur} ist die unterste der drei Ebenen und befasst sich mit der Informationstechnologie. Dazu zählt laut Wollnik die Hard- und Software sowie die inhaltlichen Strukturen (zentrale Informationsbestände, Zugriffsberechtigungen auf Informationen). Kernaufgabe dieser Ebene ist der Betrieb und die Entwicklung der Infrastrukturen.\\

Diese drei Ebenen sind hierarchisch strukturiert und stellen den jeweils übergeordneten Ebenen Dienstleistungen zur Verfügung bzw. Stellen Anforderungen an die jeweils untergeordneten Ebenen. Dieses einfache Ebenenmodell stellt auch die Grundlage für viele weitere Informationsmanagement- modelle dar, unter anderem das von Krcmar.

\subsection{Informationsmanagement nach Krcmar}
Krcmars\footnote{in welchem Werk?} Strukturierung des Informationsmanagement basiert auf dem Ebenenmodell von Wollnik, erweitert es jedoch um allgemeine Führungsaufgaben mit ebenenübergreifenden Funktionen (IT-Governance, Strategie, IT-Prozesse, IT-Personal, IT-Controlling).\\

Er gliedert das Informationsmanagement in die drei Teilbereiche Informationswirtschaft, Informationssysteme und Informations- und Kommunikationstechnik.\\

Die \textbf{Informationswirtschaft} beschäftigt sich mit dem Angebot, der Nachfrage und Verwendung von Informationen.
Die \textbf{Informationssysteme} haben das Management von Daten, Prozessen und dem Anwendungslebenszyklus zur Aufgabe.
Die \textbf{Informations- und Kommunikationstechnik} weisen die Speicherung, Verarbeitung und Kommunikation von Information als Basisfunktionalitäten auf.
