\section{Begriffsdefinition des Wortes Informationsmanagement}
Das Informationsmanagement ist ein Bestandteil der Unternehmensführung und hat planende, kontrollierende und steuernde Aufgaben sowohl im strategischen als auch im operativen Bereich zu erfüllen. Zudem soll es die Entscheidungsprozesse in den Unternehmen oder Organisationen, in denen Informationsmanagement eingesetzt wird, mit den nötigen Informationen zu versorgen. Informationen sollten im Rahmen des Informationsmanagements als Ressource angesehen werden, die im Unternehmen gesammelt, verarbeitet und genutzt werden kann.\\

Das Informationsmanagement lässt sich im Wesentlichen in drei Aufgabenbereiche unterteilen. \\

Zum einen hat es die Klärung und Planung des \textbf{Informationsbedarfs} zur Aufgabe, in der abgewägt werden muss, welche Informationen (Qualität), wann (Dringlichkeit) und in welchem Umfang (Quantität) benötigt werden.\\


Ist der Informationsbedarf geklärt, muss die \textbf{Informationsbeschaffung} geplant und organisiert werden. Hier stellt sich die Frage, wo (Ort, Quelle, Medium), wie (Werkzeuge), wann (im günstigsten Moment) und durch wen (Qualifikation, Fähigkeiten) die Informationen beschafft werden können.\\


Sind die Informationen beschafft, folgt die \textbf{Informationssicherung}, \textbf{Nutzbarmachung} und \textbf{Nutzenmehrung}. Hier müssen die Informationen aufbereitet (Aus- und Bewerten), verarbeitet (Integrieren und Kombinieren), präsentiert (vor einer entsprechenden Zielgruppe) und dokumentiert (Archivieren) werden.\\

\subsection{Begriffsdefinition Information}
Information sind
\begin{itemize}
	\item immaterielle Güter
	\item keine freien Güter
	\item beliebig zu vervielfältigen
	\item nicht abnutzbar
	\item leicht erweiterbar und verdichtbar;
	\item leicht und schnell zu transportieren.
\end{itemize}

Informationen unterliegen des weiteren einem Informationslebenszyklus \emph{(Muss weiter ausgearbeitet werden)}.