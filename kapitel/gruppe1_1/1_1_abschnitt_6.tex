\section{Anwendung des Informationsmanagements am Beispiel von Hochschulen}
Ein gut funktionierendes Qualitätsmanagement kann nur effektiv und reibungslos funktionieren, wenn es an zentraler Stelle nahe des Entscheidungsträgers positioniert und gelebt wird. Die Umsetzungsverantwortung eines ganzheitlichen Qualitätsmanagements liegt bei Hochschulen in der Regel bei der Hochschulleitung, die ihre Aufgaben im Prozess der Informationsflussoptimierung begreifen und verantworten muss.\\

Eine der grundlegendsten Besonderheiten an Hochschulen liegt in der internen Strukturierung von Verantwortlichkeiten. Die Hochschule ist unterteilt in Fachbereiche, welche geschlossen für sich arbeiten können, aber dennoch der Hochschulleitung unterstellt sind. Zusätzlich zu diesen beiden Bereichen ist noch das Präsidium zu nennen, welches insgesamt für eine effiziente Aufgabenerfüllung und Interessensvertretung der Hochschule verantwortlich ist.\footnote{Quelle 7}\\

Im Zuge der Einführung eines geordneten Qualitätsmanagements gilt es also, die Positionierung nahe der Hochschulleitung mit einer anwendungsbezogenen Platzierung innerhalb jedes Fachbereiches unter Einbeziehung des Präsidiums zu verknüpfen, um ganzheitliche Lösungen zur Realisierung eines Qualitätsmanagements zu finden und umsetzen zu können. Ein Außenvorlassen des Fachbereichs, in dem die Lösungen schließlich umgesetzt werden, ist faktisch unmöglich. Durch die Vielzahl an Entscheidungsträgern und Mitredern besteht an Hochschulen ein höherer Bedarf an Kommunkations- und Abstimmungsleistungen zwischen diesen als in anderen Institutionen und Unternehmen. Es besteht zudem die Gefahr, dass Zuständigkeiten der verschiedenen Rollen an der entsprechenden Hochschule nicht klar geregelt sind, was die Funktionsweise des Entscheidungsprozesses zwar bestenfalls nicht beeinträchtigt, dessen Ablauf allerdings sehr unsystematisch gestaltet und den Fluß des Prozesses ausbremst.\\

Neben der strukturellen Schwierigkeiten in der Aufstellung eines Qualitätsmanagements besteht eine weitere Besonderheit in der inhaltlichen Vereinheitlichung der Anforderungen der einzelnen Parteien, die im schlechtesten Fall sehr verschieden sind oder sich gar widersprechen, sodass diese für alle Bereiche zentral gültig ist.\\

Mithilfe renommierter Werkzeuge, wie z.B. der IT Balanced Scorecard, liegt es nun in der Hand des Qualitätsmanagement-Teams, die erarbeiteten Prozessstrategien und Maßnamen transparent für jeden Bereich der Hochschule einsehbar zu publizieren und alle betreffenden Personen über Änderungen zu informieren. Die Kontrolle in den Fachbereichen, ob und inwieweit die Maßnahmen zur Prozessoptimierung beitragen, darf hierbei nicht vernachlässigt werden.\footnote{Quelle 8}

\subsection{Immatrikulations- und Prüfungsamt}
In Hochschulen, bei denen ein Informationsmanagement Anwendung findet, bildet das Immatrikulations- und Prüfungsamt eine Art interne Informationszentrale, welche weitere Bereiche mit notwendigen Informationen versorgt. Betrachet an einem Beispiel bedeutet dies Folgendes: Bei Immatrikulation eines neuen Studierenden wird diesem vom Immatrikulationsamt eine Matrikelnummer zugewiesen und seine Stammdaten ins HIS eingepflegt. Nun ist es Aufgabe des Immatrikulationsamtes, das HIS zu einer Art Schnittstelle für alle wichtigen Hochschulbereiche, wie z.B. die Bibliothek, die Mensa oder auch die Verwaltung von Computerräumen, zu machen, sodass diese Bereiche via Eingabe der Matrikelnummer auf für sie wichtige Studierendendaten zugreifen können. Um den Datenschutz der Studierenden zu garantieren, wäre hierfür eine Lösung mittels individueller Rechtezuweisung für jeden Bereich denkbar.\\

Der absolut saubere und stets aktuelle Datensatz im HIS wäre nicht nur zentral für alle Hochschulbereiche verfügbar, sondern auch jederzeit auf aktuellstem Stand, sodass Redundanzen ausgeschlossen werden können. Zur Minimierung des Verwaltungsaufwandes, wäre es denkbar, bei Stammdatenänderung durch das Immatrikulationsamt eine automatisch generierte E-Mail an alle beteiligten Bereiche mit den aktualisierten Informationen über den Studierenden zu versenden, was einem ganzheitlichen Informationsmanagement entsprechen würde.\\

Auch nach außen hin stellt das HIS eine zentrale Anlaufstelle für alle wichtigen Informationen wie Raumpläne, Kontaktdaten der Lehrenden und Prüfungsmodalitäten dar. Bei Ausfall einer Veranstaltung kann dieses dort direkt publik gemacht werden. Nach der Prüfungsanmeldung im HIS kann schnell und komfortabel aus den Anmeldedaten der Studierenden ein zentraler Raumbelegungsplan erzeugt werden.\footnote{Quelle 9}\\

Bei der Notenvergabe meldet der Prüfer die Noten der Studierenden an das Prüfungsamt, welche diese in das HIS einpflegen. Die Studierenden haben nun die Möglichkeit zentral ihre Noten abzurufen. Auch die Fachbereiche, welche über die Leistungen ihrer Studierenden informiert werden sollten, können auf diese Daten zugreifen.

Die Sammlung und Bereitstellung an zentraler Stelle wie dem HIS minimiert Abstimmungsmodalitäten zwischen den verschiedenen Hochschulbereichen, reduziert den Arbeitsaufwand für die erneute Erfassung und Verwaltung der Studierendendaten in dem jeweiligen Bereich und garantiert einen stets konsistenten Datensatz.

\subsection{Bibliotheken}
Hochschulbibliotheken werden tagtäglich mit einer Menge an Informationen und Daten konfrontiert. Von deren Besitz eines EDV-Systems zur Erfassung der Ausleihe inkl. Ablauf der Fristen und Stammdaten des Studierenden kann an dieser Stelle ausgegangen werden, da die Grundfunktionalität des Bibliothekssystems ansonsten kaum gewährleistet wäre. Als weitere Basisfunktion sei die Autorisierung der Studierenden zu nennen. Bei der Ausleihe wird in Hochschulbibliotheken über das System geprüft, ob dieser Studierende durch Immatrikulation dazu berechtigt ist, an dieser Hochschule Bücher auszuleihen.\\

Im Zuge eines angewandten Informationsmanagements wäre es von Vorteil, die Stammdaten der Studierenden direkt aus dem HIS auszulesen.\\

Aufbauend auf dieses Grundsystem existieren Lösungen, die das Bibliothekswesen mittels Informationsvermittlung, -speicherung und -auswertung für zahlreiche Einsatzmöglichkeiten bereichert. Jede Hochschule sollte sich etwas Zeit nehmen, sich mit einer EDV-Lösung zu befassen, die neben der elektronischen Erschließung der Ausleihfaktoren auch Werkzeuge zur statistischen Erfassung, Messung und Bewertung der Bestandsentwicklung und des Leihverhaltens bietet. Aus diesen statistischen Daten können Rückschlüsse auf das Verhalten der Studenten gezogen und wichtige Erkenntnisse für den weiteren Bestandsaufbau gezogen werden.\footnote{Quelle 10}\\

Je nach Größe der Bibliothek ist es sinnvoll, sich grundlegend Gedanken darüber zu machen, welche Mitarbeiter für die Medienbestellung zuständig sind und wer die Entscheidungskompetenz besitzt. Eine kontinuierliche Abstimmung optimalerweise mittels zentralem Verwaltungssystem untereinander ist unumgänglich, um Doppelbestellungen zu vermeiden und das Budget möglichst gewinnbringend für die Studierenden einzusetzen.\\

Die Mitarbeiter, die für die Medienbestellungen zuständig sind, sollten sich stetig auf dem Laufenden halten, welche Neuerungen es auf dem Büchermarkt gibt, um diese Werke möglichst aktuell in den Bestand aufnehmen zu können und den Studierenden eine topaktuelle Ausleihe zu garantieren. Die Bibliotheksleitung könnte über Kooperationen mit anderen Hochschulen zum Austausch von Neuerungen oder auch zum Tausch von Dubletten nachdenken, um dem Gesamtkonzept eines gelebten Informationsmanagements gerecht zu werden.\\

Die Studierenden könnten via Newsletter oder Website der Bibliothek darüber informiert werden, welche Neuerungen in den Bücherbestand aufgenommen wurden. Ab einer gewissen Bibliotheksgröße könnte auch ein Online-Katalog angedacht werden, der das Repertoire der Bibliothek abbildet und wichtige Informationen nach außen trägt. Ohne diese zentralen Informationsplattformen wäre ein Informationsmanagement an der Hochschule überflüssig.\\

\subsection{Rechnerpools}
Die Organisation der Nutzung von Rechnerpools zieht ohne zentrales Informationsmanagement einige Probleme nach sich. Doppelbelegungen und unnötig leerstehende Computerräume sind die Folge eines fehlenden zentralen Belegungssystems.\\

Das bereits erläuterte HIS könnte um genau diese Funktion erweitert werden. Die Lehrenden können sich im HIS einen Computerraum für ihre Lehrveranstaltungen verbindlich reservieren und bei Ausfall der Veranstaltung wieder für die Allgemeinheit freigeben. Da die Raumbelegung an zentraler Stelle geschieht, ist auch hier der klare Vorteil, dass der Plan jederzeit auf aktuellem Stand ist und von jedem Lehrenden oder Studierenden eingesehen werden kann, was Verzögerungen, die bei der Suche eines geeigneten Computerraums auf herkömmlichem Wege, eliminiert.
